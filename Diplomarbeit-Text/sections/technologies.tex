%\section{Foo}
%\setauthor{Stefan Schwammal}
%\lipsum[5-12]

%\section{Bar}
%\setauthor{Susi Schwammal}
%\lipsum[12-18]

%\subsection{Deeper}
%Nicht mehr im Inhaltsverzeichnis.

%\subsubsection{Deepest}
%Vermeide mich.

\section{React Native}
React native allows for the development of cross platform mobile applications using javascript/typescript and the react framework.
For this project React Native was chosen as the framework for the mobile application
because of the author's experience with typescript as well as its cross platform compatability
which is important because the school staff does not have a uniform phone operating system.

\subsection{Expo}
Expo was used as a development environment for its ease of use and quick development cycle.
Expo allows you to use your own phone to test your application without the need for an emulator like Android Studio.
It also enables the easy use of the camera via the expo-camera package which was especially useful for this project.

\section{Visual Studio Code}
The author used Visual Studio code as his editor of choice for this project because it was set up very quickly without any complications.

\section{Communication}
The communication between the document understanding model running os a flask server and the mobile application is done via WebSockets
because the model needs to accept incoming images of the application form and return the extracted data for validation.

Was kann man da alles noch dazu schreiben? Access? Programming language?